\documentclass{abnt} 

\input{Base}


%%%%% Dados para criação da capa e folha de rosto %%%%
\autor{Guilherme e Victor}
\titulo{Justificativa}
\orientador{Thalita}
\comentario{}
\instituicao{}
\local{Campos do Jordão}
\data{\today}

\begin{document}
	%\bibliographystyle{plain}
	% Para utilizar o formato padrão de capa da ABNT, substituí o comando \maketitle pelo comando \capa.

	%\capa
	
	%\folhaderosto
	
	%\sumario 
	
	%\listadetabelas
	
	%\listadefiguras

	Guilherme e Victor.

	\section{Justificativa}



	Hoje em dia, o transporte veicular se tornou algo de suma importância para o âmbito
	econômico e social de nossa sociedade, impactando tanto positivamente quanto negativa-
	mente. 
	Além de agilizar o transporte de pessoas, materiais e animais, também gera um
	custo para as empresas manterem estes veículos. Se um veículo percorrer uma menor rota,
	a empresa diminui custos, como, por exemplo, combustível, manutenção e tempo. 

	Cada vez as cidades estão ficando maiores e a complexidade rodoviária cresce, dificultando a
	análise da melhor rota a se percorrer. Esse problema tende a ficar mais complexo, conforme o aumento de
	veículos que a empresa possui.
	
	A solução proposta é desenvolver um \textit{software} que resolva o problema do MTSP(\textit{Multiple Traveling Salesman Problem}).
	Esta solução será integrada com o Google Maps, podendo ser usado para calcular rota de vans escolares e entregas de mercadorias, com diversos veículos,
	diminuindo custo e tempo.





\end{document}
