\documentclass{beamer}
\usetheme{Berlin}
%Arquivo com os principais pacotes usados e suas descrições.

%%%%%%%%%%%%%%%%%%%%%%%%%%%%%%%%%%%%%%%%%
% 			Idiomas e Acentos			%
%%%%%%%%%%%%%%%%%%%%%%%%%%%%%%%%%%%%%%%%%
\usepackage[brazil]{babel} % Habilita o uso do idioma português do brasil (PT-BR).
\usepackage[T1]{fontenc} 
%\usepackage{fontspec} % Habilita maior variedade de acentos. Pode ser necessario adicionar outros pacotes.
\usepackage{lmodern} % Habilita o uso da font Latin Modern.


%%%%%%%%%%%%%%%%%%%%%%%%%%%%%%%%%%%%%%%%%
% 				TABELAS					%
%%%%%%%%%%%%%%%%%%%%%%%%%%%%%%%%%%%%%%%%%
%\usepackage{tabulary} % Cria tabelas mais facilmente.
%\usepackage{booktabs} % Melhora o visual das tabelas.
%\usepackage[table]{xcolor} % Pacote de cor pra as tabelas.
%\usepackage{caption} % Melhora as legendas de imagens, tabela etc.

%%%%%%%%%%%%%%%%%%%%%%%%%%%%%%%%%%%%%%%%%
% 				IMAGENS					%
%%%%%%%%%%%%%%%%%%%%%%%%%%%%%%%%%%%%%%%%%
%\usepackage{graphicx} % Facilita a inserção de imagens.


%%%%%%%%%%%%%%%%%%%%%%%%%%%%%%%%%%%%%%%%%
% 			CÓDIGO FONTE				%
%%%%%%%%%%%%%%%%%%%%%%%%%%%%%%%%%%%%%%%%%
%Documentação de código fonte.
\usepackage{listings}


%%%%%%%%%%%%%%%%%%%%%%%%%%%%%%%%%%%%%%%%%
% 	Símbolos e Caracteres Matemáticos	%
%%%%%%%%%%%%%%%%%%%%%%%%%%%%%%%%%%%%%%%%%
\usepackage{amsmath}
\usepackage{amssymb}
\usepackage{amsfonts}
%\usepackage{mathspec} %Habilita o uso das fontes e dos caracteres matematicos.


%%%%%%%%%%%%%%%%%%%%%%%%%%%%%%%%%%%%%%%%%
%				ABNT					%
%%%%%%%%%%%%%%%%%%%%%%%%%%%%%%%%%%%%%%%%%
%\usepackage[alf]{abntcite} % Ordena as referencias em ordem alfabética.
\usepackage{url} %Facilita o uso de url. Pode-se usar o comando \url{...}.


%%%%%%%%%%%%%%%%%%%%%%%%%%%%%%%%%%%%%%%%%
% 			Configurações				%
%%%%%%%%%%%%%%%%%%%%%%%%%%%%%%%%%%%%%%%%%
%\captionsetup{justification=centering,labelfont=bf} %Formata a legenda das figuras.
%\graphicspath{{../imgs/}} %Define o diretorio padrão para buscar as imagens da apresentação.  
%\setromanfont[Ligatures=TeX]{Crimson}
%\defaultfontfeatures{Scale=MatchLowercase, Mapping=tex-tex}

%%%%%%%%%%%%%%%%%%%%%%%%%%%%%%%%%%%%%%%%%
%				BEAMER					%
%%%%%%%%%%%%%%%%%%%%%%%%%%%%%%%%%%%%%%%%%
%Define algumas configurações que serão validas para todo o documento.  
\setbeamertemplate{section in toc}[sections numbered]
\setbeamertemplate{subsection in toc}[subsections numbered]
\setbeamertemplate{background canvas}[vertical shading][bottom=blue!3,top=blue!7]
\setbeamertemplate{caption}[numbered]

\author{Guilherme A. de Macedo \\
        Victor H. Carlquist da Silva
}

\title{\titulo{DESENVOLVIMENTO DE UM ALGORITMO GENÉTICO HÍBRIDO PARA A SOLUÇÃO DO PROBLEMA DOS MÚLTIPLOS CAIXEIROS VIAJANTES}}
\date{\today}

\begin{document}

    \frame{
        \titlepage
    }
	
	\frame{\titlepage {Introdução}
	    \begin{itemize}
	        \item Transporte veicular;
	        \item Gera despesas;
	        \item Análise da melhor rota a se percorrer;
	        \item Algoritmos Genéticos (GA);
	        \item Algoritmos de Dijkstra;
	        \item Algoritmo Genético Híbrido (AGH);
	        \item Transporte Escolar em Santo Antônio do Pinhal.
	    \end{itemize}
	}
	
	\frame{\titlepage {Revisão da Literatura}
	    \begin{itemize}
	        \item Problema do Caixeiro Viajante (\textit{Traveling Salesman Problem} - TSP);
	        \item Algoritmos Genéticos;
	        \item Operadores Genéticos;
	        \item Operador de Mapeamento Parcial (\textit{PMX});
	        \item Algoritmos Genéticos e o Problema dos Caixeiros Viajantes;
	        \item O Algoritmo de Dijkstra.
	    \end{itemize}
	}
	
	\frame{\titlepage {O Algoritmo Desenvolvido}
	    \begin{itemize}
	        \item Two-part Chromossome;
	        \item Cruzamento;
	        \item Mutação;
	    \end{itemize}
	}	
	
	\frame{\titlepage {Inicialização e Execução do Algoritmo}
        $$ A *a = new GA(12, 3, 0.75f, 3000, depot[12, 13, 14], 12, objectives [0,1,2,3,4,5,6,7,8,9,10,11], 15, bairros[\ref{tb1}]); $$
        
        $$ Chromosome *x = a->evolution(); $$
	}
	
	\frame{\titlepage {Resultados das Implementações}
		\begin{table}[h]
		\caption{\label{para}Tempo de execução - Modelo real.}
		\rowcolors{1}{gray!20}{white}
		\begin{tabular}{|l|c|c|c|c|c|} \hline 
			\textbf{Bairros} & 
			\textbf{Nº Caixeiros} & 
			\textbf{\% mutação} & 
			\textbf{Gerações} & 
			\textbf{Objetivos} & 
			\textbf{Execução}\\ \hline
			15	   	&  	3 	&   0.2 	&  100        &	 12	     &  0m0.518s \\ \hline 
		\end{tabular}
	\end{table}
	}
	
	\frame{\titlepage {Resultados das Implementações}
        \begin{table}[h]
            \caption{\label{tempo}Tempo de execução.}
            \rowcolors{1}{gray!20}{white}
            \begin{tabular}{|l|c|c|c|c|c|} \hline
                \textbf{Bairros}  & \textbf{Nº Caixeiros}  & \textbf{\% mutação}    & \textbf{Gerações} & \textbf{Objetivos} & \textbf{Execução}\\ \hline
                2000        &           10  &   0.2         &  100          &    20         &  2m1.719s         \\ \hline 
                1000        &           10  &   0.2         &  100          &   200         &  7m34.427s        \\ \hline 
                1000        &           10  &   0.2         &  100          &    20         &  1m13.621s        \\ \hline 
                1000        &           10  &   0.2         &  100          &    10         &  0m52.927s        \\ \hline
                1000        &           1   &   0.2         &  100          &    10         &  15m37.259s       \\ \hline 
                100         &           10  &   0.2         &  100          &    20         &  0m0.862s         \\ \hline 
                100         &           10  &   0.2         &  1000         &    20         &  0m20.694s        \\ \hline 
                100         &           1   &   0.2         &  100          &    10         &  0m2.728s         \\ \hline 
                100         &           10  &   0.2         &  100          &    90         &  0m6.562s         \\ \hline 
            \end{tabular}
        \end{table}
	}
	
	\frame{\frametitle{Resultados das Implementações}
        \begin{table}[h]
            \rowcolors{1}{gray!20}{white}
            \centering
            \caption{\label{tb4} Execução do AGH}
            \begin{tabular}{|c|c|c|} \hline
			    \textbf{Execução}    & \textbf{Score}          &      \textbf{Cromossomo} \\ \hline
                1	        &    87	         &   10 0 1 9 4 11 6 5 8 3 2 7 2 3 7 \\ \hline 
                2	        &    91	         &   8 0 4 5 6 3 1 7 2 11 9 10 2 3 7 \\ \hline
                3	        &    91	         &   8 5 0 4 10 3 2 1 7 9 11 6 2 3 7 \\ \hline
                4	        &    89	         &   11 4 0 8 6 5 1 10 9 7 3 2 2 3 7 \\ \hline
                5	        &    85	         &   4 3 9 2 0 5 10 11 7 8 6 1 2 3 7 \\ \hline
                6	        &    88	         &   6 7 1 3 5 10 0 11 9 4 8 2 2 3 7 \\ \hline
                7	        &    89	         &   3 1 2 6 5 7 10 4 0 11 9 8 2 3 7 \\ \hline
                8	        &    93	         &   1 6 5 11 10 8 9 2 7 0 4 3 2 3 7 \\ \hline
                9	        &    85	         &   4 0 3 2 1 11 9 5 8 7 6 10 2 3 7 \\ \hline
                10	        &    93	         &   9 4 8 5 7 11 3 10 6 1 0 2 2 3 7 \\ \hline
            \end{tabular}
        \end{table} 
    }
	
	\frame{
		\frametitle{Conclusões e Trabalhos Futuros}
		 
	}
\end{document}
		
