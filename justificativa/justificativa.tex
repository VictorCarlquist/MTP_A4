\documentclass{abnt} 

%Arquivo com os principais pacotes usados e suas descrições.

%%%%%%%%%%%%%%%%%%%%%%%%%%%%%%%%%%%%%%%%%
% 			Idiomas e Acentos			%
%%%%%%%%%%%%%%%%%%%%%%%%%%%%%%%%%%%%%%%%%
\usepackage[brazil]{babel} % Habilita o uso do idioma português do brasil (PT-BR).
\usepackage[T1]{fontenc} 
%\usepackage{fontspec} % Habilita maior variedade de acentos. Pode ser necessario adicionar outros pacotes.
\usepackage{lmodern} % Habilita o uso da font Latin Modern.


%%%%%%%%%%%%%%%%%%%%%%%%%%%%%%%%%%%%%%%%%
% 				TABELAS					%
%%%%%%%%%%%%%%%%%%%%%%%%%%%%%%%%%%%%%%%%%
%\usepackage{tabulary} % Cria tabelas mais facilmente.
%\usepackage{booktabs} % Melhora o visual das tabelas.
%\usepackage[table]{xcolor} % Pacote de cor pra as tabelas.
%\usepackage{caption} % Melhora as legendas de imagens, tabela etc.

%%%%%%%%%%%%%%%%%%%%%%%%%%%%%%%%%%%%%%%%%
% 				IMAGENS					%
%%%%%%%%%%%%%%%%%%%%%%%%%%%%%%%%%%%%%%%%%
%\usepackage{graphicx} % Facilita a inserção de imagens.


%%%%%%%%%%%%%%%%%%%%%%%%%%%%%%%%%%%%%%%%%
% 			CÓDIGO FONTE				%
%%%%%%%%%%%%%%%%%%%%%%%%%%%%%%%%%%%%%%%%%
%Documentação de código fonte.
\usepackage{listings}


%%%%%%%%%%%%%%%%%%%%%%%%%%%%%%%%%%%%%%%%%
% 	Símbolos e Caracteres Matemáticos	%
%%%%%%%%%%%%%%%%%%%%%%%%%%%%%%%%%%%%%%%%%
\usepackage{amsmath}
\usepackage{amssymb}
\usepackage{amsfonts}
%\usepackage{mathspec} %Habilita o uso das fontes e dos caracteres matematicos.


%%%%%%%%%%%%%%%%%%%%%%%%%%%%%%%%%%%%%%%%%
%				ABNT					%
%%%%%%%%%%%%%%%%%%%%%%%%%%%%%%%%%%%%%%%%%
%\usepackage[alf]{abntcite} % Ordena as referencias em ordem alfabética.
\usepackage{url} %Facilita o uso de url. Pode-se usar o comando \url{...}.


%%%%%%%%%%%%%%%%%%%%%%%%%%%%%%%%%%%%%%%%%
% 			Configurações				%
%%%%%%%%%%%%%%%%%%%%%%%%%%%%%%%%%%%%%%%%%
%\captionsetup{justification=centering,labelfont=bf} %Formata a legenda das figuras.
%\graphicspath{{../imgs/}} %Define o diretorio padrão para buscar as imagens da apresentação.  
%\setromanfont[Ligatures=TeX]{Crimson}
%\defaultfontfeatures{Scale=MatchLowercase, Mapping=tex-tex}

%%%%%%%%%%%%%%%%%%%%%%%%%%%%%%%%%%%%%%%%%
%				BEAMER					%
%%%%%%%%%%%%%%%%%%%%%%%%%%%%%%%%%%%%%%%%%
%Define algumas configurações que serão validas para todo o documento.  
\setbeamertemplate{section in toc}[sections numbered]
\setbeamertemplate{subsection in toc}[subsections numbered]
\setbeamertemplate{background canvas}[vertical shading][bottom=blue!3,top=blue!7]
\setbeamertemplate{caption}[numbered]



%%%%% Dados para criação da capa e folha de rosto %%%%
\autor{Guilherme e Victor}
\titulo{Justificativa}
\orientador{Thalita}
\comentario{}
\instituicao{}
\local{Campos do Jordão}
\data{\today}

\begin{document}
	%\bibliographystyle{plain}
	% Para utilizar o formato padrão de capa da ABNT, substituí o comando \maketitle pelo comando \capa.

	%\capa
	
	%\folhaderosto
	
	%\sumario 
	
	%\listadetabelas
	
	%\listadefiguras

	Guilherme e Victor.

	\section{Justificativa}



	Hoje em dia, o transporte veicular se tornou algo de suma importância para o âmbito
	econômico e social de nossa sociedade, impactando tanto positivamente quanto negativa-
	mente. 
	Além de agilizar o transporte de pessoas, materiais e animais, também gera um
	custo para as empresas manterem estes veículos. Se um veículo percorrer uma menor rota,
	a empresa diminui custos, como, por exemplo, combustível, manutenção e tempo. 

	Cada vez as cidades estão ficando maiores e a complexidade rodoviária cresce, dificultando a
	análise da melhor rota a se percorrer. Esse problema tende a ficar mais complexo, conforme o aumento de
	veículos que a empresa possui.
	
	A solução proposta é desenvolver um \textit{software} que resolva o problema do MTSP(\textit{Multiple Traveling Salesman Problem}).
	Esta solução será integrada com o Google Maps, podendo ser usado para calcular rota de vans escolares e entregas de mercadorias, com diversos veículos,
	diminuindo custo e tempo.





\end{document}
