\documentclass[final,5p,times,twocolumn]{elsarticle}
\usepackage[T1]{fontenc}

%% Use the option review to obtain double line spacing
%% \documentclass[preprint,review,12pt]{elsarticle}

%% Use the options 1p,twocolumn; 3p; 3p,twocolumn; 5p; or 5p,twocolumn
%% for a journal layout:
%% \documentclass[final,1p,times]{elsarticle}
%% \documentclass[final,1p,times,twocolumn]{elsarticle}
%% \documentclass[final,3p,times]{elsarticle}
%% \documentclass[final,3p,times,twocolumn]{elsarticle}
%% \documentclass[final,5p,times]{elsarticle}
%% \documentclass[final,5p,times,twocolumn]{elsarticle}

%% if you use PostScript figures in your article
%% use the graphics package for simple commands
%% \usepackage{graphics}
%% or use the graphicx package for more complicated commands
%% \usepackage{graphicx}
%% or use the epsfig package if you prefer to use the old commands
%% \usepackage{epsfig}

%% The amssymb package provides various useful mathematical symbols
%% \usepackage{amssymb}
%% The amsthm package provides extended theorem environments
%% \usepackage{amsthm}

%% The lineno packages adds line numbers. Start line numbering with
%% \begin{linenumbers}, end it with \end{linenumbers}. Or switch it on
%% for the whole article with \linenumbers after \end{frontmatter}.
%% \usepackage{lineno}

%% natbib.sty is loaded by default. However, natbib options can be
%% provided with \biboptions{...} command. Following options are
%% valid:

%%   round  -  round parentheses are used (default)
%%   square -  square brackets are used   [option]
%%   curly  -  curly braces are used      {option}
%%   angle  -  angle brackets are used    <option>
%%   semicolon  -  multiple citations separated by semi-colon
%%   colon  - same as semicolon, an earlier confusion
%%   comma  -  separated by comma
%%   numbers-  selects numerical citations
%%   super  -  numerical citations as superscripts
%%   sort   -  sorts multiple citations according to order in ref. list
%%   sort&compress   -  like sort, but also compresses numerical citations
%%   compress - compresses without sorting
%%
%% \biboptions{comma,round}

% \biboptions{}


\journal{Nuclear Physics B}

\begin{document}

\begin{frontmatter}

%% Title, authors and addresses

%% use the tnoteref command within \title for footnotes;
%% use the tnotetext command for the associated footnote;
%% use the fnref command within \author or \address for footnotes;
%% use the fntext command for the associated footnote;
%% use the corref command within \author for corresponding author footnotes;
%% use the cortext command for the associated footnote;
%% use the ead command for the email address,
%% and the form \ead[url] for the home page:
%%
%% \title{Title\tnoteref{label1}}
%% \tnotetext[label1]{}
%% \author{Name\corref{cor1}\fnref{label2}}
%% \ead{email address}
%% \ead[url]{home page}
%% \fntext[label2]{}
%% \cortext[cor1]{}
%% \address{Address\fnref{label3}}
%% \fntext[label3]{}

\title{Análise do uso de Algoritmos Genéticos na resolução do problema dos Múltiplos Caixeiros Viajantes (\textit{mTSP - Multiple Traveling Salesman Problem})}

%% use optional labels to link authors explicitly to addresses:
%% \author[label1,label2]{<author name>}
%% \address[label1]{<address>}
%% \address[label2]{<address>}

\author{Victor Hugo Carlquist da Silva, Guilherme Augusto de Macedo}

\address{Campos do Jordão/SP}

\begin{abstract}
%% Text of abstract
Lorem ipsum dolor sit amet, consectetuer adipiscing elit, sed diam nonummy nibh euismod tincidunt 
ut laoreet dolore magna aliquam erat volutpat. Ut wisi enim ad minim veniam, quis nostrud exerci 
tation ullamcorper suscipit lobortis nisl ut aliquip ex ea commodo consequat. Duis autem vel eum 
iriure dolor in hendrerit in vulputate velit esse molestie consequat, vel illum dolore eu feugiat 
nulla facilisis at vero eros et accumsan et iusto odio dignissim qui blandit praesent luptatum 
zzril delenit augue duis dolore te feugait nulla facilisi.

\end{abstract}

\begin{keyword}
%% keywords here, in the form: keyword \sep keyword
Algoritmos Genéticos \sep 
Múltiplos Caixeiros Viajantes \sep 
Otimização \sep 
Roteirização \sep
Grafos
%% MSC codes here, in the form: \MSC code \sep code
%% or \MSC[2008] code \sep code (2000 is the default)

\end{keyword}

\end{frontmatter}

%%
%% Start line numbering here if you want
%%
%\linenumbers

%%%%%%%%%%%%%%%%%%%%%%%%%%%%%%%%%%%%%%%%%%%%%%%%%%%%%%%%%%%%%%%%%%%%%%%%%%%%%%%%%%%%%%%%%%%%%%%%%%%%%%%%%%%%%%
%%%%%%%%%%%%%%%%%%%%%%%%%%%%%%%%%%%%%%%%%%%%%%%%%%%%%%%%%%%%%%%%%%%%%%%%%%%%%%%%%%%%%%%%%%%%%%%%%%%%%%%%%%%%%%

%% Título: Análise do uso de Algoritmos Genéticos na resolução do problema dos Múltiplos Caixeiros Viajantes (\textit{mTSP - Multiple Traveling Salesman Problem})

%% main text
\section{Introdução}
	%% Formulação simples e clara do tema da pesquisa e apresentação reduzida do estado da arte do problema
		%% Explicar sobre algoritmos genéticos e sobre o TSP; e também sobre mTSP
		
		Algoritmos Genéticos são inspirados no princípio Darwiniano da evolução das espécies e na genética \cite{0004-pdf}. 
		São utilizados para achar o melhor caminho e têm como característica básica o fato de que o mesmo algoritmo, 
		executado mais de uma vez, pode resultar em resultados diferentes, mas sempre próximo da solução ótima. AGs são utilizados para resolverem problemas de otimização, utilizando o método probabilístico.
	
		O Problema do Caixeiro Viajante (\textit{TSP - Traveling Salesman Problem}) é bastante conhecido e visa escolher o 
		melhor caminho ao percorrer todas as arestas de um grafo partindo de um determinado vértice.

		%% Justificativa
		%% Dizer que resolver o problema do TSP é fácil, mas com o mTSP é muito mais difícil.
		%% Por isso utilização de AG.
		
		Ao utilizar AGs para resolver o problema do Caixeiro Viajante observou-se que esse tipo de algorimo é menos eficiente 
		do que soluções feitas com algoritmos menos arrojados.
		
		Quando pensamos em Múltiplos Caixeiros Viajantes, a complexidade do problema aumenta de maneira significante justificando, 
		assim, o uso de AGs em sua solução.
		
	%% Objetivos
		%%Esperamos explicar de forma clara por que os AG são mais eficientes para resolver esse tipo de problema
		
	%% Exposição da metodologia
		%% Resolveremos o problema dos mTSP com e sem AG, mediremos os resultados, entenderemos o que aconteceu e 
		%% mostraremos os resultados
	%% Situação de tempo e espaço em que o problema é estudado
		%% Faremos isso utilizando a biblioteca tal no SO Linux com a seguinte configuração...
		Será implementada em linguagem \textit{C}, utilizando o compilador \textit{GCC}, padrão no \textit{Linux}, 
		algumas soluções de AG aplicadas na resolução do \textit{mTSP}. Com base nessas implementações, será 
		realizada uma análise de desempenho sobre os resultados obtidos. Com isso, será possível demonstrar quais 
		técnicas são mais eficazes na resolução desse tipo de problema.
		
	
%% Desenvolvimento - Explicação: tornar evidente o que estava implícito, obscuro ou complexo.
	
\section {Algoritmos Genéticos}
	%% Algotimos Genéticos explicados mais a fundo do que na introdução.

	%% Genética
	Em seu reconhecido livro A Origem das Espécies, Charles Darwin propôs que os organismos vivos se adaptam através 
	da seleção natural. Seleção natural nada mais é do que a preservação das condições favoráveis e a eliminação das 
	variações nocivas. Essa seleção natural é obtida através do cruzamento entre os indivíduos, gerando assim novos
 	indivíduos mais adaptados.

	%% Breve Histórico dos AGs
	Com base no princípio de Charles Darwin, John Holland começou a desenvolver pesquisas sobre o tema e em 1975 
	publicou uma obra intitulada "Adaptation in Natural and Artificial Systems" hoje considerada a bíblia dos AGs.
	Em 1980, um aluno Holland consegue a primeira aplicação industrial de AGs, desde então esses tipos de algoritmos vem 
	sendo utilizados em técnicas de busca e otimização.

	%% Analogia entre AGs e Evolução das espécies
	Na biologia, o código genético é a identidade do indivíduo e esses são representados pelos cromossomas. 
	Na computação, a analogia pode ser dada pelo cromossoma sendo uma estrutura de dados que após submetido 
	ao processo evolucionário, processo esse constituído por vários ciclos, gera indivíduos cada vez mais 
	aptos.

	Dessa forma, AGs são algoritmos probabilísticos que fornecem um mecanismo de busca paralela e adaptativa. O indivíduo melhor adaptado terá uma probabilidade maior de se reproduzir \cite{0004-pdf}.

	%% Características dos AGs
	O objetivo dos AGs é gerar indivíduos cada vez mais aptos, isso não significa que com a execução do algoritmo 
	diversas vezes para a mesma solução, o indivíduo final será o mesmo.

\section {Problema do Caixeiro Viajante (\textit{Traveling Salesman Problem} - TSP}
	%% TSP explicados mais a fundo do que na introdução.
	%Dentre os problemas de roteirização de veículos, o problema do Caixeiro Viajante (TSP) é o que possuí .
	O objetivo do TSP é encontrar em um grafo G = ( N , A ) o caminho de menor custo, 
	de forma que todos os vértices sejam visitados uma única vez \cite{0006-pdf}.
	
	Nesse problema, o objetivo é o caixeiro viajante visitar todos os seus clientes (nós) uma única vez 
	e voltar ao ponto de partida.

\section {Problema dos Múltiplos Caixeiros Viajantes (\textit{Multiple Traveling Salesman Problem } – mTSP}
	%% mTSP explicados mais a fundo do que na introdução.
	O Problema dos Múltiplos Caixeiros Viajantes (mTSP) é uma extensão do problema do caixeiro viajante, 
	porém com múltiplos roteiros e múltiplos caixeiros viajantes\cite{0006-pdf}. 
	

\section {Algoritmos Genéticos e o Problema dos Caixeiros Viajantes}
	%% AG versus TSP
	
	A utilização de AGs motrou-se mais eficiente na resolução do mTSP do que na resolução do TSP,  
	levando-se em conta um maior tempo de processamento para o caso mais complexo \cite{0005-pdf}.

	A solução apresentada em \cite{0006-pdf} propõe resolver os problemas de roteirização de 
	veículos com entregas fracionadas, problema clássico de roteirização de veículos e com 
	frota heterogênia criando o algoritmo de roteirização de veículos com frota heterogênea, 
	restrições de janelas de tempo e entregas fracionadas(\textit{Heterogeneous Fleet Vehicle 
	Routing Problem with Time Windows and Split Deliveries}(HFVRPTWSD)) utilizando Algoritmo 
	Genético(AG).

	Também é possível calcular as rotas de múltiplos veículos utilizando AG para igualar o tempo 
	de espera de encomendas de clientes, sendo que a variável "menor tempo da rota" não é levada 
	em consideração \cite{0005-pdf}.	

\section {Contextualização - Necessidade de se achar o menor caminho}
	%% AG e mTSP
	Hoje em dia o transporte veicular tornou-se algo de suma importância, 
	impactando tanto positivamente quanto negativamente na sociedade e no meio ambiente. 
	
	Além de agilizar o transporte de pessoas, materiais e animais, 
	o transporte veicular também gera despesas com combustível, manutenção, etc. 
	Se um veículo percorrer uma menor rota, a empresa diminui custos, com, por exemplo, 
	combustível, manutenção e tempo de entrega.

	Cada vez as cidades estão ficando maiores e a complexidade rodoviária cresce, dificultando a
	análise da melhor rota a se percorrer. Esse problema tende a ficar mais complexo conforme 
	aumenta a quantidade de veículos que a empresa possui. Com isso, surge a necessidade de criar 
	\textit{softwares} cada vez mais robustos que resolvam o problema de roteirização de veículos.

	
\section {Implementações de Algoritmos Genéticos na Resolução do Problema dos Múltiplos Caixeiros Viajantes}

\section {Resultados das Implementações}
		
	%% Demonstrar: aplicar a argumentação apropriada à natureza do trabalho;
		%% Apresentação dos resultados obtidos.

\section{Conclusões e Trabalhos Futuros}
	%% Afirmação sintética da ideia central do trabalho e de pormenores apresentados no texto
	%% Comentários e consequência da pesquisa, aberturas novas
		% É isso!

%%%%%%%%%%%%%%%%%%%%%%%%%%%%%%%%%%%%%%%%%%%%%%%%%%%%%%%%%%%%%%%%%%%%%%%%%%%%%%%%%%%%%%%%%%%%%%%%%%%%%%%%%%%%%%
%%%%%%%%%%%%%%%%%%%%%%%%%%%%%%%%%%%%%%%%%%%%%%%%%%%%%%%%%%%%%%%%%%%%%%%%%%%%%%%%%%%%%%%%%%%%%%%%%%%%%%%%%%%%%%

%% The Appendices part is started with the command \appendix;
%% appendix sections are then done as normal sections
%% \appendix

%% \section{}
%% \label{}

%% References
%%
%% Following citation commands can be used in the body text:
%% Usage of \cite is as follows:
%%   \cite{key}          ==>>  [#]
%%   \cite[chap. 2]{key} ==>>  [#, chap. 2]
%%   \citet{key}         ==>>  Author [#]

%% References with bibTeX database:

\bibliographystyle{model1a-num-names}
\bibliography{refs.bib}

\end{document}

%%
%% End of file `elsarticle-template-1a-num.tex'.
