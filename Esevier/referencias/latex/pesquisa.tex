\documentclass[12pt,openright,a4paper]{abntex2}
\usepackage{fontspec}
\usepackage[alf]{abntex2cite}
\usepackage{lastpage}
%\usepackage{ifsp-cjo}

\titulo{}
\autor{Guilherme Augusto de Macedo, Victor Hugo Carquist da Silva}
\data{\today}
\instituicao{IFSP - Instituto Federal de Educação, Ciência e Tecnologia de São Paulo - Campus Campos do Jordão}
\orientador{Thalita Biazzuz Veronese}
\preambulo{... trabalho feito para o curso de Tecnologia e Análise de Desenvolvimento de Sistemas do \imprimirinstituicao}
\tipotrabalho{Pesquisa}
\local{Campos do Jordão - SP}

\makeatletter
\hypersetup{
    pdftitle={\@title},
    pdfauthor={\@author},
    pdfsubject={\imprimirpreambulo},
    pdfkeywords={PALAVRAS}{CHAVES}{EM}{PORTUGUES},
    pdfcreator={LaTeX with abnTeX2},
    colorlinks=true,
    linkcolor=blue,
    citecolor=blue,
    urlcolor=blue
}
\makeatother

\begin{document}

\imprimircapa
\imprimirfolhaderosto

\tableofcontents

\newpage

\begin{resumo}
	Hoje em dia, o transporte veicular se tornou algo de suma importância para o âmbito
	econômico e social de nossa sociedade, impactando tanto positivamente quanto negativa-
	mente. 
	Além de agilizar o transporte de pessoas, materiais e animais, também gera um
	custo para as empresas manterem estes veículos. Se um veículo percorrer uma menor rota,
	a empresa diminui custos, como, por exemplo, combustível, manutenção e tempo. 

	Cada vez as cidades estão ficando maiores e a complexidade rodoviária cresce, dificultando a
	análise da melhor rota a se percorrer. Esse problema tende a ficar mais complexo, conforme o aumento de
	veículos que a empresa possui.
	
	A solução proposta é desenvolver um \textit{software} que resolva o problema do MTSP (\textit{Multiple Traveling Salesman Problem}).
	Esta solução será integrada com o Google Maps, podendo ser usado para calcular rota de vans escolares e entregas de mercadorias, com diversos veículos,
	diminuindo custo e tempo.
    \vspace{\onelineskip}
    \noindent
    %\textbf{Palavras-chaves}: latex. abntex. editoração de texto.
\end{resumo}

\textual

\part{Parte Inicial}

\chapter{Resumo das Referências}

\section{Algoritmos Genéticos}
Artigo de introdução básica sobre o assunto. \cite{0001-pdf}
\\

\section{O problema do entregador viajante}
Problema do entregador viajante. A diferença entre o problema clássico 
do caixeiro viajante está no fato de que, nesse caso, o que foi levado 
em conta foi a rapidez da entrega, não o melhor caminho a ser percorrido. 
Pode ajudar na pesquisa nos introduzindo ao conceito de algoritmo genético. \cite{0002-pdf}
\\

\section{Desenvolvimento de um Algoritmo Genético}
Apresentação sobre desenvolvimento de um algoritmo genético. Fala como representar, 
decodificar, avaliar, operadores, técnicas e parâmetros. Pode nos auxiliar no 
desenvolvimento do nosso próprio algoritmo. \cite{0003-pdf}
\\

\section{Algoritmos Genéticos: princípios e aplicações}
Apostila contendo informações básicas sobre algoritmos genéticos. Faz uma analogia 
sobre a teoria Darwiniana e a teoria dos algoritmos genéticos. Pode nos ajudar a 
entender o assunto. \cite{0004-pdf}
\\

\section{O problema de roteirização periódica de veículos}
O problema de roteirização periódica de veículos. Já entrando no tema do trabalho que trata de MTSP. 
Esse artigo fala sobre o problema de roteirização periódica de veículos visando melhorar os custos 
gerados pela má organização dos roteiros. Há uma referência ao problema de roteirização periódica 
serem definidos como problemas de múltiplos carteiros viajantes com restrições de capacidade e outras 
restrições... Ajuda a dar uma visão geral sobre o tema dos Múltiplos Caixeiros Viajantes. \cite{0005-pdf}
\\

\section{Scatter search para Problemas de Roteirização de Veículos com Frota Heterogênea, Janelas de Tempo e Entregas}
Problema de roteirização com frota heterogênea, restrições de janelas de tempo e entregas fracionadas. 
É uma fusão dos problemas de roteirização com frota heterogênica (HFVRP), com janelas de tempo (VRPTW)
e com entregas fracionadas (VRPSD). Pode nos ajudar dando-nos uma ideia de como esses problemas trabalham 
separadamente.\cite{0006-pdf}
\\

\section{Multiple pickup and delivery traveling salesman problem with last-in-first-out loading and distance constraints}
Problema do Caixeiro viajante com mútiplos coletas e entregas com carregamento de fila e restrições 
de distância. Trata do problema do caixeiro viajante extendido por adicionar dois fatores: o uso 
de múltiplos veículos e a limitação na distância que cada veículo pode percorrer, ambos ocorrendo 
ao mesmo tempo. Pode nos ajudar a entender melhor o problema do mTSP. \cite{0007-pdf}
\\

\section{Agoritmos Genéticos - Introdução}
Apresenta os algoritmos genéticos de forma simplificada. \cite{0008-pdf}

\postextual

\bibliography{refs}

\end{document}
