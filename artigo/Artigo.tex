\documentclass[final,5p,times,twocolumn]{elsarticle}
\usepackage[T1]{fontenc}

%% Use the option review to obtain double line spacing
%% \documentclass[preprint,review,12pt]{elsarticle}

%% Use the options 1p,twocolumn; 3p; 3p,twocolumn; 5p; or 5p,twocolumn
%% for a journal layout:
%% \documentclass[final,1p,times]{elsarticle}
%% \documentclass[final,1p,times,twocolumn]{elsarticle}
%% \documentclass[final,3p,times]{elsarticle}
%% \documentclass[final,3p,times,twocolumn]{elsarticle}
%% \documentclass[final,5p,times]{elsarticle}
%% \documentclass[final,5p,times,twocolumn]{elsarticle}

%% if you use PostScript figures in your article
%% use the graphics package for simple commands
%% \usepackage{graphics}
%% or use the graphicx package for more complicated commands
%% \usepackage{graphicx}
%% or use the epsfig package if you prefer to use the old commands
%% \usepackage{epsfig}

%% The amssymb package provides various useful mathematical symbols
%% \usepackage{amssymb}
%% The amsthm package provides extended theorem environments
%% \usepackage{amsthm}

%% The lineno packages adds line numbers. Start line numbering with
%% \begin{linenumbers}, end it with \end{linenumbers}. Or switch it on
%% for the whole article with \linenumbers after \end{frontmatter}.
%% \usepackage{lineno}

%% natbib.sty is loaded by default. However, natbib options can be
%% provided with \biboptions{...} command. Following options are
%% valid:

%%   round  -  round parentheses are used (default)
%%   square -  square brackets are used   [option]
%%   curly  -  curly braces are used      {option}
%%   angle  -  angle brackets are used    <option>
%%   semicolon  -  multiple citations separated by semi-colon
%%   colon  - same as semicolon, an earlier confusion
%%   comma  -  separated by comma
%%   numbers-  selects numerical citations
%%   super  -  numerical citations as superscripts
%%   sort   -  sorts multiple citations according to order in ref. list
%%   sort&compress   -  like sort, but also compresses numerical citations
%%   compress - compresses without sorting
%%
%% \biboptions{comma,round}

% \biboptions{}


\journal{Nuclear Physics B}

\begin{document}

\begin{frontmatter}

%% Title, authors and addresses

%% use the tnoteref command within \title for footnotes;
%% use the tnotetext command for the associated footnote;
%% use the fnref command within \author or \address for footnotes;
%% use the fntext command for the associated footnote;
%% use the corref command within \author for corresponding author footnotes;
%% use the cortext command for the associated footnote;
%% use the ead command for the email address,
%% and the form \ead[url] for the home page:
%%
%% \title{Title\tnoteref{label1}}
%% \tnotetext[label1]{}
%% \author{Name\corref{cor1}\fnref{label2}}
%% \ead{email address}
%% \ead[url]{home page}
%% \fntext[label2]{}
%% \cortext[cor1]{}
%% \address{Address\fnref{label3}}
%% \fntext[label3]{}

\title{Análise do uso de Algoritmos Genéticos na resolução do problema dos Múltiplos Caixeiros Viajantes (\textit{mTSP - Multiple Traveling Salesman Problem})}

%% use optional labels to link authors explicitly to addresses:
%% \author[label1,label2]{<author name>}
%% \address[label1]{<address>}
%% \address[label2]{<address>}

\author{Victor Hugo Carlquist da Silva, Guilherme Augusto de Macedo}

\address{Campos do Jordão/SP}

\begin{abstract}
%% Text of abstract
Lorem ipsum dolor sit amet, consectetuer adipiscing elit, sed diam nonummy nibh euismod tincidunt 
ut laoreet dolore magna aliquam erat volutpat. Ut wisi enim ad minim veniam, quis nostrud exerci 
tation ullamcorper suscipit lobortis nisl ut aliquip ex ea commodo consequat. Duis autem vel eum 
iriure dolor in hendrerit in vulputate velit esse molestie consequat, vel illum dolore eu feugiat 
nulla facilisis at vero eros et accumsan et iusto odio dignissim qui blandit praesent luptatum 
zzril delenit augue duis dolore te feugait nulla facilisi.

\end{abstract}

\begin{keyword}
%% keywords here, in the form: keyword \sep keyword
Algoritmos Genéticos \sep 
Múltiplos Caixeiros Viajantes \sep 
Otimização \sep 
Roteirização \sep
Grafos
%% MSC codes here, in the form: \MSC code \sep code
%% or \MSC[2008] code \sep code (2000 is the default)

\end{keyword}

\end{frontmatter}

%%
%% Start line numbering here if you want
%%
%\linenumbers

%%%%%%%%%%%%%%%%%%%%%%%%%%%%%%%%%%%%%%%%%%%%%%%%%%%%%%%%%%%%%%%%%%%%%%%%%%%%%%%%%%%%%%%%%%%%%%%%%%%%%%%%%%%%%%
%%%%%%%%%%%%%%%%%%%%%%%%%%%%%%%%%%%%%%%%%%%%%%%%%%%%%%%%%%%%%%%%%%%%%%%%%%%%%%%%%%%%%%%%%%%%%%%%%%%%%%%%%%%%%%

%% Título: Análise do uso de Algoritmos Genéticos na resolução do problema dos Múltiplos Caixeiros Viajantes (\textit{mTSP - Multiple Traveling Salesman Problem})

%% main text
\section{Introdução}
	


	%% Formulação simples e clara do tema da pesquisa e apresentação reduzida do estado da arte do problema
		%% Explicar sobre algorítmos genéticos e sobre o TSP; e também sobre mTSP
	%% Justificativa
		%% Dizer que resolver o problema do TSP é fácil, mas com o mTSP é muito mais difícil.
		%% Por isso utilização de AG.
	%% Objetivos
		%% Esperamos explicar de forma clara por que os AG são mais eficientes para resolver esse tipo de problema
	%% Exposição da metodologia
		%% Resolveremos o problema dos mTSP com e sem AG, mediremos os resultados, entenderemos o que aconteceu e 
		%% mostraremos os resultados
	%% Situação de tempo e espaço em que o problema é estudado
		%% Faremos isso utilizando a biblioteca tal no SO Linux com a seguinte configuração...

	
	%Today, transport vehicle  become very important for economy and social sphere, impacting in positive and negative ways.
	%Also it makes faster the transport of people, materials and animals, although companies have to pay for it. If a vehicle travels in a minor route, companies decreases cost with fuel, maintenance and time.

	%Each passing day the cities are getting bigger and roads complexity grows proportionally, difficulting the analisy to get a better route to take.
	%this problem tends to get more complex with the increase of vehicles.

	%The solution propound is to developer a software for solve the \textit{Multiple Traveling Salesman Problem}(MTSP).
	----------------------------------------------------  


	Hoje em dia, o transporte veicular se tornou algo de suma importância para o âmbito
	econômico e social de nossa sociedade, impactando tanto positivamente quanto negativa-
	mente na sociedade e no meio ambiente. 
	Além de agilizar o transporte de pessoas, materiais e animais, também gera um
	custo para as empresas manterem estes veículos. Se um veículo percorrer uma menor rota,
	a empresa diminui custos, como, por exemplo, combustível, manutenção e tempo. 

	Cada vez as cidades estão ficando maiores e a complexidade rodoviária cresce, dificultando a
	análise da melhor rota a se percorrer. Esse problema tende a ficar mais complexo, conforme o aumento de
	veículos que a empresa possui. Com isso, surge a necessidade de criar \textit{software} cada vez mais robustos e que resolvam o problema de roteirização de veículos.

	Utilizando Algoritmo Genético(AG), que é um algoritmo baseado na teoria da evolução das espécies de C. Darwin \cite{0004-pdf}, mostrou-se  que os AGs não são eficientes na solução do Problema do Caixeiro Viajante (\textit{Travelling Salesman Problem}(TSP)) mas mostrou-se ser capaz de encontrar o melhor resultado no problema dos Múltiplos Caixeiros Viajantes (\textit{Multiple Travelling Salesman Problem}(mTSP)), mas com um tempo maior de processamento \cite{0005-pdf}. 	

	A solução apresentada em \cite{0006-pdf} propõe resolver os problemas de roteirização de veículos com entregas fracionadas, problema clássico de roteirização de veículos e com frota heterogênia criando o algoritmo de roteirização de veículos com frota
heterogênea, restrições de janelas de tempo e entregas fracionadas(\textit{Heterogeneous Fleet Vehicle Routing Problem with Time Windows and Split Deliveries}(HFVRPTWSD)) utilizando Algoritmo Genético(AG).

	Também é possível calcular as rotas de multíplos veículos utilizando AG para igualar o tempo de espera de encomendas de clientes, sendo que a variável "menor tempo da rota" não é levada em consideração \cite{0005-pdf}.
	
	
	

	%A solução proposta é desenvolver um \textit{software} que resolva o problema do MTSP(\textit{Multiple Traveling Salesman Problem}).
	%Esta solução será integrada com o Google Maps, podendo ser usado para calcular rota de vans escolares e entregas de mercadorias, com diversos veículos, diminuindo custo e tempo.

\section{Desenvolvimento}
	%% Explicação: tornar evidente o que estava implícito, obscuro ou complexo.
		%% Algotimos Genéticos explicados mais a fundo do que na introdução.
		%% TSP explicados mais a fundo do que na introdução.
		%% mTSP explicados mais a fundo do que na introdução.
	%% Discutir:   comparar várias posições que se entrechocam dialeticamente
		%% TSP versus mTSP
		%% mTSP sem AG
		%% mTSP com AG
	%% Demonstrar: aplicar a argumentação apropriada à natureza do trabalho;
		%% Apresentação dos resultados obtidos.



\section{Conclusão e Trabalhos Futuros}
	%% Afirmação sintética da ideia central do trabalho e de pormenores apresentados no texto
	%% Comentários e consequência da pesquisa, aberturas novas
		% É isso!

%%%%%%%%%%%%%%%%%%%%%%%%%%%%%%%%%%%%%%%%%%%%%%%%%%%%%%%%%%%%%%%%%%%%%%%%%%%%%%%%%%%%%%%%%%%%%%%%%%%%%%%%%%%%%%
%%%%%%%%%%%%%%%%%%%%%%%%%%%%%%%%%%%%%%%%%%%%%%%%%%%%%%%%%%%%%%%%%%%%%%%%%%%%%%%%%%%%%%%%%%%%%%%%%%%%%%%%%%%%%%

%% The Appendices part is started with the command \appendix;
%% appendix sections are then done as normal sections
%% \appendix

%% \section{}
%% \label{}

%% References
%%
%% Following citation commands can be used in the body text:
%% Usage of \cite is as follows:
%%   \cite{key}          ==>>  [#]
%%   \cite[chap. 2]{key} ==>>  [#, chap. 2]
%%   \citet{key}         ==>>  Author [#]

%% References with bibTeX database:

\bibliographystyle{model1a-num-names}
\bibliography{refs.bib}

\end{document}

%%
%% End of file `elsarticle-template-1a-num.tex'.
