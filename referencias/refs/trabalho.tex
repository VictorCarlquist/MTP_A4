John Holland foi quem começou a desenvolver as primeiras pesquisas sobre Algoritmos Genéticos. Holland publicou em 
1975 o livro \textit{Adaptation in Natural and Artificial Systems}, hoje considerado o livro de maior influência nos
Algoritmos Genéticos (AG). 

Os algoritmos genéticos(AG) se inspiram na evolução biológica e na teoria de C. Darwin. O principio da seleção natural que privilegia o indivíduo com melhor adaptação ao meio ambiente.

As principais características do processamento dos AGs são:
\begin{itemize}
	\item Problema a ser otimizado;
	\item Representação das Soluções de Problema;
	\item Decodificação do Cromossoma;
	\item Avaliação;
	\item Seleção;
	\item Operadores Genéticos;
	\item Inicialização da População.
\end{itemize}


Um dos fatores principais que define a eficiência do AG é a escolha correta da aptidão dos indivíduos que o AG gera.

O AG hoje em dia é muito utilizado em roteirização de veículos, porém, ele pode ser utilizados em diversos problemas por ser um algoritmo "flexível", podendo ser utilizado com outros métodos.

