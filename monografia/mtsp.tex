\documentclass{abnt} 

%Arquivo com os principais pacotes usados e suas descrições.

%%%%%%%%%%%%%%%%%%%%%%%%%%%%%%%%%%%%%%%%%
% 			Idiomas e Acentos			%
%%%%%%%%%%%%%%%%%%%%%%%%%%%%%%%%%%%%%%%%%
\usepackage[brazil]{babel} % Habilita o uso do idioma português do brasil (PT-BR).
\usepackage[T1]{fontenc} 
%\usepackage{fontspec} % Habilita maior variedade de acentos. Pode ser necessario adicionar outros pacotes.
\usepackage{lmodern} % Habilita o uso da font Latin Modern.


%%%%%%%%%%%%%%%%%%%%%%%%%%%%%%%%%%%%%%%%%
% 				TABELAS					%
%%%%%%%%%%%%%%%%%%%%%%%%%%%%%%%%%%%%%%%%%
%\usepackage{tabulary} % Cria tabelas mais facilmente.
%\usepackage{booktabs} % Melhora o visual das tabelas.
%\usepackage[table]{xcolor} % Pacote de cor pra as tabelas.
%\usepackage{caption} % Melhora as legendas de imagens, tabela etc.

%%%%%%%%%%%%%%%%%%%%%%%%%%%%%%%%%%%%%%%%%
% 				IMAGENS					%
%%%%%%%%%%%%%%%%%%%%%%%%%%%%%%%%%%%%%%%%%
%\usepackage{graphicx} % Facilita a inserção de imagens.


%%%%%%%%%%%%%%%%%%%%%%%%%%%%%%%%%%%%%%%%%
% 			CÓDIGO FONTE				%
%%%%%%%%%%%%%%%%%%%%%%%%%%%%%%%%%%%%%%%%%
%Documentação de código fonte.
\usepackage{listings}


%%%%%%%%%%%%%%%%%%%%%%%%%%%%%%%%%%%%%%%%%
% 	Símbolos e Caracteres Matemáticos	%
%%%%%%%%%%%%%%%%%%%%%%%%%%%%%%%%%%%%%%%%%
\usepackage{amsmath}
\usepackage{amssymb}
\usepackage{amsfonts}
%\usepackage{mathspec} %Habilita o uso das fontes e dos caracteres matematicos.


%%%%%%%%%%%%%%%%%%%%%%%%%%%%%%%%%%%%%%%%%
%				ABNT					%
%%%%%%%%%%%%%%%%%%%%%%%%%%%%%%%%%%%%%%%%%
%\usepackage[alf]{abntcite} % Ordena as referencias em ordem alfabética.
\usepackage{url} %Facilita o uso de url. Pode-se usar o comando \url{...}.


%%%%%%%%%%%%%%%%%%%%%%%%%%%%%%%%%%%%%%%%%
% 			Configurações				%
%%%%%%%%%%%%%%%%%%%%%%%%%%%%%%%%%%%%%%%%%
%\captionsetup{justification=centering,labelfont=bf} %Formata a legenda das figuras.
%\graphicspath{{../imgs/}} %Define o diretorio padrão para buscar as imagens da apresentação.  
%\setromanfont[Ligatures=TeX]{Crimson}
%\defaultfontfeatures{Scale=MatchLowercase, Mapping=tex-tex}

%%%%%%%%%%%%%%%%%%%%%%%%%%%%%%%%%%%%%%%%%
%				BEAMER					%
%%%%%%%%%%%%%%%%%%%%%%%%%%%%%%%%%%%%%%%%%
%Define algumas configurações que serão validas para todo o documento.  
\setbeamertemplate{section in toc}[sections numbered]
\setbeamertemplate{subsection in toc}[subsections numbered]
\setbeamertemplate{background canvas}[vertical shading][bottom=blue!3,top=blue!7]
\setbeamertemplate{caption}[numbered]



%%%%% Dados para criação da capa e folha de rosto %%%%
\autor{Victor Hugo Carlquist da Silva}
\titulo{MTSP}
\orientador{Thalita Biazzuz Veronese}
\comentario{Trabalho apresentado a Profa. Thalita Biazzuz Veronese, na disciplina de Metodologia de Pesquisa 
			 no $4^{a}$ módulo do curso de Tecnologia em Análise e Desenvolvimento de Sistemas no IFSP-CJO.}
\instituicao{Instituto Federal de Educação, Ciência e Tecnologia de São Paulo -- \textit{campus} Campos do Jordão}
\local{Campos do Jordão}
\data{\today}

\begin{document}
	%\bibliographystyle{plain}
	% Para utilizar o formato padrão de capa da ABNT, substituí o comando \maketitle pelo comando \capa.

	\capa
	
	\folhaderosto
	

	
	\sumario 
	
	\listadetabelas
	
	\listadefiguras
	
	\chapter{Introdução}

	

	\section{Justificativa}
		
		Hoje em dia o transporte veicular tornou-se algo de suma importância, 
		impactando tanto positivamente quanto negativamente na sociedade e no meio ambiente, principalmente na econômia. 
		
		Além de agilizar o transporte de pessoas, materiais e animais, 
		o transporte veicular também gera despesas com combustível, manutenção, etc. 
		Se um veículo percorrer uma menor rota, a empresa diminui custos, com, por exemplo, 
		combustível, manutenção e tempo de entrega.

		Cada vez as cidades estão ficando maiores e a complexidade rodoviária cresce, dificultando a
		análise da melhor rota a se percorrer. Esse problema tende a ficar mais complexo conforme 
		aumenta a quantidade de veículos que a empresa possui. Com isso, surge a necessidade de criar 
		\textit{softwares} cada vez mais robustos que resolvam o problema de roteirização de veículos.

	\section{Metodologia}

		A metodologia de pesquisa para o desenvolvimento deste trabalho esta classificada a seguir:
		\begin{itemize}
			\item Natureza: Aplicada;
			\item Quantos aos objetivos: Exploratória;
			\item Procedimentos técnicos: Bibliográfica, documental e experimental;
		\end{itemize}
	
		O objetivo é desenvolver um \textit{software} que encontre a rota ótima, ou quase ótima, para múltiplos caixeiros viajantes. O experimento será empírico, modificando o número de vertices do digrafo será possível estudar o desempenho das rotas e o tempo de execução.

		Esta pesquisa tem como objetivo mostrar o uso de Algorítmos Genéticos na resolução do problemas dos múltiplos caixeiros viajantes(\textit{Multiple Traveling Salesman Problem} - MTSP)
	
	\section{Algoritmos genéticos}

		Segundo \cite{0008-pdf}, os Algoritmos Genéticos(AG) são técnicas de procura e optimização baseados em mecanismos de seleção natural. 

		Nas décadas de 60 e 70, John Holland e seus colegas da Universidade de Michigan criaram modelos para estudar o processo de adaptação dos seres vivos. Holland realizou diversas pesquisas e em 1975 publicou o seu livro intitulado \textit{Adaptation in Natural and Artificial System}. Hoje, este livro é considerado um dos mais importantes sobre Algoritmos Genéticos \cite{0001-pdf}.

		Em AG, o cromossoma é uma estrutura de dados que armazena uma possível solução de um problema. Sendo que cada indivíduo tem um cromossoma, os indivíduos são cruzados gerando novos indivíduos, conforme a população cresce, surgem indivíduos cada vez mais aptos, sendo que um deles será o mais apto, contendo no seu cromossoma a solução do problema.

		O AG possuí alguns parâmetros que devem ser levados em consideração \cite{0001-pdf}:

		\begin{itemize}
			\item \textit{Tamanho da população: } Uma população pequena deixará o lento o desempenho, pois terá um pequeno conjunto para a busca de solução do problema. Já uma população muito grande pode afetar o desenpenho do algoritmo;
			\item \textit{Taxa de Cruzamento: } Quanto maior a taxa, mais rapidaemnte novos indivíduos serão introduzidos na população. Mas se a taxa for muito alta, pode-se eleminar indivíduos aptos. Com a taxa de cruzamento pequena, o algoritmo se torna lento;
			\item \textit{Taxa de Mutação: } Com uma alta taxa de mutação o algoritmo fica aleatório, mas com uma baixa taxa previne que os indivíduos sejam os mesmos.
		\end{itemize}


		\subsection{Operadores de cruzamento}

			Os operadores de cruzamento definem como ocorrerá o cruzamento entre dois indivíduos.

		\subsubsection{\textit{Partially-mapped crossover} - PMX} 

			O operador de cruzamento de mapeamento parcial seleciona copia os genes do pai e substiu alguns genes de um outro pai, como a \textbf{figura ~\ref{pmx}} demonstra:

			%Figura pmx 0012.pdf pg75

		\subsubsection{Operador de Cruzamento \textit{Edge Recombination} (ERX)}

			Este método é muito utilizado para o problema do caixeiro viajante por priorizar a adjacência dos vértices do grafo.

			Os passos para a geração dos filhos é a seguinte:

			\begin{enumerate}
				\item Criar uma lista de arestas de ambos os pais;
				\item Deverá ser criada uma lista para cada cidade com todas as cidades conectada a ela e a, pelo menos, um gene do pai
			\end{enumerate}


		\section{Problema do Caixeiro Viajante}

		O problema do Caixeiro Viajante(PCV) consiste em estabeler uma rota que passe por cada vértice do grafo apenas uma vez e retorne ao vértice de partida. O número de rotas possíveis pode ser expressa por (n-1)!, sendo n o número de vértices.
		O problema PCV é classificado como \textit{NP-Hard}, ou seja, não existe algoritmo com limitação polinominal capaz de resolvẽ-lo.\cite{0010-pdf}


	\chapter{História}
	\section{História}
	
	\chapter{Conclusão}
	
	
	\nocite{R7}
	\nocite{fisio}
	\nocite{olhar}
	
	\bibliography{referencias}

\end{document}
